%----------------------------------------------------------------------------------------
%	SECTION TITLE
%----------------------------------------------------------------------------------------

\cvsection{Formation}

%----------------------------------------------------------------------------------------
%	SECTION CONTENT
%----------------------------------------------------------------------------------------

\begin{cventries}

%------------------------------------------------

\cventry
{Fréquence du master en ingénierie informatique} % Degree
{\href{https://tecnico.ulisboa.pt/en/}{Instituto Superior Técnico}} % Institution
{Lisbonne, Portugal} % Location
{septembre. 2016 - présent} % Date(s)
{  % Description(s) bullet points
\begin{cvitems}
\item {Expertise pour les systèmes intelligents en lien avec des champs thématiques variés : l’analyse intelligente des données, l’imagerie, les Interfaces Homme Machine (IHM), l’intelligence artificielle.}
\item{Thèmatiques: Intelligence artificielle , Systèmes intelligentes, Systémes multi-agents, Intelligence artificielle distribué, Apprentissage automatique}
\item{Disciplines:}
\begin{itemize}
\item Systèmes d'appui aux décisions
\item Logique algorithmique
\item Systèmes parallèles et distribués
\item Représentation des connaissances
\end{itemize}
\end{cvitems}
}


\cventry
{Licence ingénierie informatique} % Degree
{Faculté des Sciences et Technologies, \href{https://www.uc.pt/}{Université de Coimbra}} % Institution
{Coimbra, Portugal} % Location
{septembre. 2010 - juin 2014} % Date(s)
{  % Description(s) bullet points
\begin{cvitems}
\item {Acquisition de principes, théories, méthodes et pratiques de l'ingénierie informatique, des logiciels et des systèmes de réseau, avec une plus grande spécialisation dans les sujets concernant la technologie informatique, la théorie de l'ingénierie, les mathématiques, les réseaux et de la programmation.}
\item{Disciplines:}
\begin{itemize}
\item Programmation avancée et structures dynamiques
\item Réseaux
\item Analyse des algorithmes
\item Programmation linéaire
\item Systèmes d’information
\item Logique
\item Anglais
\item Génie logiciel
\item Bases de données
\item Anlyse mathématique I et II
\item Algèbre linéaire et géométrie analytique
\item Introduction à la programmation
\item Algorithmique et Structures de Données
\item Introduction à la Physique Moderne
\item Gestion de l'innovation
\item Introduction `a l'Intelligence Artificielle
\item système d'information
\item Introduction à la statistique
\item Architecture des ordinateurs
\item Génie logiciel
\item Algorithmes et structures de données
\item Analyse de données
\item théories de la computation
\item Programmation Orientée Objet
\item Introduction aux réseaux de communication
\item Compilateurs : principes, techniques et outils
\item multimédia
\item systèmes d'information
\item Systèmes Distribués
\item Computation graphique
\item Systemes Operationnels
\end{itemize}
\item{Principales technologies: Python 2.7, C, C++, Java, J2EE, Linux, UML, Base de dados Oracle, MySQL, Adobe Flash, Assembly, OpenGL, HTML5 }
\end{cvitems}
}


\cventry
{Baccalauréat informatique obtenu en 2010} % Degree
{\href{http://www.brotero.pt/}{Lycée Avelar Brotero}} % Institution
{Coimbra, Portugal} % Location
{septembre. 2007 - juin. 2010} % Date(s)
{  % Description(s) bullet points
\begin{cvitems}
\item {Acquisition des connaissances et des compétences en programmation, en génie logiciel, en systèmes matériels, en bases de données, en télécommunications, de même qu'en mathématiques, en économie et en administration.}
\item{Principales technologies:}
\begin{itemize}
\item Pascal
\item Visual Basic
\item Linux
\end{itemize}
\end{cvitems}
}


%------------------------------------------------

\end{cventries}

\pagebreak