%----------------------------------------------------------------------------------------
%	SECTION TITLE
%----------------------------------------------------------------------------------------

\cvsection{Expérience professionnelle}

%----------------------------------------------------------------------------------------
%	SECTION CONTENT
%----------------------------------------------------------------------------------------

\begin{cventries}

%------------------------------------------------

\cventry
{Ingénieur de software} % Job title
{\href{https://www.fccn.pt/en/}{FCCN}, Fondation nationale pour la science et informatique} % Organization
{Lisbonne, Portugal} % Location
{novembre 2016 - présent} % Date(s)
{ % Description(s) of tasks/responsibilities
\begin{cvitems}
\item {Ingénieur software dans l’archive de la Web portugaise. Responsable pour l’indexation des contenus récoltés de la web. Développement d’un protótipe de recherche d’images à propos des pages archivées. Implémentation et réactivation du SonarQube pour le contrôle de la qualité du logiciel.}
\item {Technologies utilisées: Java 8, eclipse, Spring boot, Json, Sublime, SonarQube, Hadoop, Apache Lucene, Solr, Github, CentOS, Python, Shell Scripting.}
\end{cvitems}
}

%------------------------------------------------

\cventry
{Ingénieur de software} % Job title
{Findmore} % Organization
{Lisbonne, Portugal} % Location
{novembre 2015 - novembre 2016} % Date(s)
{ % Description(s) of tasks/responsibilities
\begin{cvitems}
\item { \textbf{\href{http://www.universia.pt/}{UNIVERSIA} (Éducation)}}
\begin{itemize}
\item {Création d’un prótotique d’une application pour la vente de billets de football online.
Responsable pour le développement Full-Stack de l’application web dans une équipe de deux personnes. L’application a été développé en Angularjs + Spring Boot. Contrôle des versions: github. Contrôle de qualité du code: SonarQubue. Gestion du flux du développement: Trello.}
\item {Technologies utilisées: AngularJS, HTML5, Bootstrap, CSS3, Spring boot, Java, github, Trello, SonarQube, Eclipse IDE}
\end{itemize}
\item { \textbf{\href{https://www.santandertotta.pt/pt_PT/Particulares.html}{Santander Totta} (Banque)}}
\begin{itemize}
\item {Responsable pour le développement des fonctionnalités d’une application web de génération KPIs à partir des données consolidées du Reporting ibérique. Notamment: implémentation d’un éditeur de diagrammes fonctionnels dont la sortie des données sont des commandes SQL.}
\item {Technologies utilisées: Java, Primefaces, Javascript, Jquery, Oracle SQL}
\end{itemize}
\item { \textbf{\href{https://www.realnetworks.com/}{Real Networks} (Telécomunications)}}
\begin{itemize}
\item {Résolution des tickets évolutives. Modification de divers applications Web. Notamment la création des filtres de sécurité pour prévenir différents types d'attaques, la création de fonctionnalités dans une application Web, particulièrement une jukebox de musiques d'attente personnalisées à la choix du client pour les appels en attente, intégration du système Oauth dans une application Web.}
\item {Technologies utilisées: Java, JSP, Spring Batch, Maven, HTML5, CSS3,  Solaris}
\end{itemize}
\end{cvitems}
}

%------------------------------------------------    

\cventry
{Ingénieur de software} % Job title
{Coswichted} % Organization
{Lisbonne, Portugal} % Location
{février 2015 - novembre 2015} % Date(s)
{ % Description(s) of tasks/responsibilities
\begin{cvitems}
\item { \textbf{\href{https://www.nos.pt/}{NOS} (Télécommunications)}}
\begin{itemize}
\item {Maintenance évolutive et corrective de projets multi-domaines (Marketing, Operations, Billing). 
Responsable des api (PRO*c et PL/SQL) pour la maintenance générique de chaque client. Création d’un système d’archivage électronique de factures (Java).}
\item {Technologies utilisées: Pro*C, Java, Oracle, SQL developer, Unix}
\end{itemize}
\end{cvitems}
}

%------------------------------------------------

\cventry
{Ingénieur de software} % Job title
{Altran Portugal} % Organization
{Lisbonne, Portugal} % Location
{septembre 2014 - février 2015} % Date(s)
{ % Description(s) of tasks/responsibilities
\begin{cvitems}
\item { \textbf{\href{https://www.iefp.pt/}{IEFP} (Institut de l'emploi et de la formation professionnelle)}}
\begin{itemize}
\item {Maintenance évolutive et corrective. Développement d’une api PL/SQL de support aux équipes de facturation. Développement des webservices SOAP, notamment pour le site officiel d’Institut de l'emploi et de la formation professionnelle (IEFP).}
\item{Technologies utilisées: PL/SQL, PL/SQL Developer, Oracle 10g, SQL Developer, Java, Eclipse, SOAP, XML, Soapui, Linux}
\end{itemize}
\end{cvitems} 
}

%------------------------------------------------

\cventry
{Stagiaire Développeur} % Job title
{\href{http://www.edp.pt/pt/Pages/homepage.aspx}{EDP} Comercial (Énergétique)} % Organization
{Coimbra, Portugal} % Location
{juillet 2013 - novembre 2013} % Date(s)
{ % Description(s) of tasks/responsibilities
\begin{cvitems}
\item {Implémentation d’une nouvelle plateforme Web pour la gestion des offres du marché non réglementé contenant des options de gestion des offres, lister/éditer/supprimer, et des options d’upload des fichiers excel import/export de toutes ces offres, et où le back-end  était le responsable de la maintenance des données. Cette plateforme avait deux perfiles: administrateur et utilisateur.}
\item{Technologies utilisées: PHP, Bootstrap, jquery, Phpmyadmin, Notepad++, Mysql}
\end{cvitems}
}

%------------------------------------------------

\cventry
{Stagiaire en contexte de formation} % Job title
{\href{https://www.ipn.pt/}{Instituto Pedro Nunes} (Incubateur d'entreprises)} % Organization
{Coimbra, Portugal} % Location
{mai 2010 - juillet 2010} % Date(s)
{ % Description(s) of tasks/responsibilities
\begin{cvitems}
\item {Implémentation d’une application de tickets où les usagers peuvent accompagner l’état des demandes et communiquer par chat avec les administrateurs.}
\item{Technologies utilisées: PHP, CSS, Phpmyadmin, Notepad++, Mysql, Linux }
\end{cvitems}
}

%------------------------------------------------

\cventry
{Stagiaire en contexte de formation} % Job title
{\href{https://www.ipn.pt/}{Instituto Pedro Nunes} (Incubateur d'entreprises)} % Organization
{Coimbra, Portugal} % Location
{mai 2009 - juillet 2009} % Date(s)
{ % Description(s) of tasks/responsibilities
\begin{cvitems}
\item {Maintenance des réseaux de l’institut. Installation et configuration des serveurs.}
\item{Technologies utilisées: Linux(Fedora, Ubuntu, Debian, OpenBSD), Servidor DNS, serveur Email, Firewall}
\end{cvitems}
}

%------------------------------------------------



\end{cventries}
