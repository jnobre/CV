%----------------------------------------------------------------------------------------
%	SECTION TITLE
%----------------------------------------------------------------------------------------

\cvsection{Formação}

%----------------------------------------------------------------------------------------
%	SECTION CONTENT
%----------------------------------------------------------------------------------------

\begin{cventries}

%------------------------------------------------

\cventry
{Mestrando em Engenharia Informática e de Computadores} % Degree
{Instituto Superior Técnico} % Institution
{Lisboa, Portugal} % Location
{Setembro. 2016 - } % Date(s)
{  % Description(s) bullet points
\begin{cvitems}
\item {Formação avançada em diversas áreas próximas da Inteligência Artificial de modo a promover as metodologias para construção de Sistemas Inteligentes. Ênfase especial para as áreas de investigação como Sistemas Multi-Agente, Inteligência Artificial Distribuída, Aprendizagem Automática.}
\item{Cadeiras: Sistema de Apoio à decisão, Procura e Planeamento, Algoritmos para Lógica Computacional}
\
\end{cvitems}
}


\cventry
{Licenciatura Engenharia Informática} % Degree
{Departamento Engenharia Informática  Faculdade Ciências e Tecnologias (Universidade de Coimbra)} % Institution
{Coimbra, Portugal} % Location
{Setembro. 2010 - Junho. 2014} % Date(s)
{  % Description(s) bullet points
\begin{cvitems}
\item {Adquisição de principios básicos, teorias, métidos e práticas da Engenharia Informática e da Computação de forma a poder dominar todos os conceitos para a sua intervenção na sociedade. Obtive competências para articular de forma harmoniosa a complementaridade entre análise e síntese, que interiorizem as dinâmicas da resolução de problemas, que apreendam as lógicas e as práticas da gestão de informação e que perspectivem o exercício dessas competências segundo abordagens de planeamento organizacional centradas na construção de valor para o cliente e na transformação da complexidade em desempenho. }
\item{Cadeiras favoritas: Sistemas Distribuidos, Sistema Operativo, Compiladores, Laboratórias de Programação Avançada, Inteligência Artificial}
\item{Principais ferramentas utilizadas: Python 2.7, C, C++, Java, J2EE, Linux, UML, Base de dados Oracle, MySQL, Adobe Flash, Assembly, OpenGL }
\
\end{cvitems}
}


\cventry
{Técnico de Informática e Programação de Sistemas Informátios} % Degree
{Escola Secundária Avelar Brotero} % Institution
{Coimbra, Portugal} % Location
{Setembro. 2007 - Junho. 2010} % Date(s)
{  % Description(s) bullet points
\begin{cvitems}
\item {O técnico de gestão e programação de sistemas informáticos é o profissional qualificado apto a realizar, de forma autónoma ou integrado numa equipa, actividades de concepção, especificação, projecto, implementação, avaliação, suporte e manutenção de sistemas informáticos e de tecnologias de processamento e transmissãode dados e informações }
\item{Cadeiras favoritas: Programação}
\item{Principais ferramentas utilizadas: Pascal, Visual Basic, Linux, Modelagem de Redes Informáticas}
\
\end{cvitems}
}


%------------------------------------------------

\end{cventries}